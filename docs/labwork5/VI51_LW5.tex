%----------------------------------------
% IT IS RECOMMENDED TO USE AUTOLATEX FOR
% COMPILING THIS DOCUMENT.
% http://www.arakhne.org/autolatex
%----------------------------------------

\documentclass[article,english,nodocumentinfo]{multiagentfrreport}

% The TeX code is entering with UTF8
% character encoding (Linux and MacOS standards)
\usepackage[utf8]{inputenc}
\usepackage{fancyhdr}
\usepackage{../common/sarl-listing}

\graphicspath{{imgs/auto/},{imgs/},{../common/}}

\declaredocument{VI51 Lab Work \#5}{Boids and Ants}{UTBM-INFO-VI51-LW5}

\addauthorvalidator*[St\'ephane Galland]{St{\'e}phane}{Galland}{Teacher}

\updateversion{5.0}{\makedate{03}{05}{2015}}{First release on Github}{\upmpublic}

\Set{mafr_contact_name}{\phdname*{St\'ephane}{Galland}}
\Set{mafr_contact_email}{stephane.galland@utbm.fr}
\Set[french]{mafr_contact_phone}{03~84~58~34~18}
\Set[english]{mafr_contact_phone}{+33 384~583~418}

\gdef\skeletonName{\texttt{\mbox{LW5\_VI51\_skeleton\string.jar}}}

\begin{document}

\section{Goal of this Lab Work Session}

The goal of this lab work session is to write the behaviors of boids, and ants.

You shall learn: 
\begin{itemize}
\item the boids behavior.
\item the ant behavior.
\item the environment model for a ant colony simulation.
\end{itemize}

\section{First Step: prepare the development environment}

In this section, you will find the tasks to do for preparing your development environment.

\subsection{Installation of the Eclipse Tools}

\begin{emphbox}
Recommended version of SARL : 0.3.1 \\
Recommended version of Janus : 2.0.3.1
\end{emphbox}

\begin{enumerate}
\item Download the \Emph{Eclipse product} that contains the compilation tools for the SARL programming language : \url{http://www.sarl.io}
\item Uncompress the Eclipse product for SARL.
\item Download the \Emph{development version} of the Janus agent execution platform : \url{http://www.janusproject.io}
\item Do not uncompress the Jar file of the Janus platform.
\item Launch the downloaded Eclipse product.
\item Open the wizard for creating a SARL project, with the menu: \\
	\texttt{> File > New > Project > SARL > Project}
\item Enter the name of the project.
\item Click on the tab with the name \code{Libraries}.
\item In the list of the libraries, remove the \code{SARL Libraries}.
\item In the list of the libraries, add the download file of the Janus platform as an external Jar file.
\item Click on "Finish", the SARL project should be created.
\item You must ensure that the configuration of your SARL project is correct (still a bug):
	\begin{enumerate}[a]
	\item Open the dialog of the properties of the SARL project by clicking on: \\
		\texttt{Right click on project > Properties > SARL > Compiler > Output Folder}
	\item Check if the field ``Directory'' is set to a source folder that is existing in your SARL project. If not change the property with): \\
		\texttt{src/main/generated-sources/xtend}
	\end{enumerate}
\end{enumerate}

You Eclipse is now ready for the lab work session. You should now install the code skeleton provided by the teachers.

\subsection{Installation of the Code Skeleton}

The teachers provide a code skeleton that should be completed by you for terminating the tasks related to this lab work session.
The steps to follow for installing the code skeleton are:
\begin{enumerate}
\item Download from the Moodle of UTBM (\url{http://moodle.utbm.fr}) the file with the name \skeletonName.
\item Open the wizard for importing the code skeleton into the SARL project: \\
	\texttt{Right click on project > Import > General > Archive File}
\item Select on the local file system the downloaded file of the code skeleton; and click on ``Finish''.
\item The source folders of your project shall contains SARL and Java code. \\
	Some errors are appearing since they are related to the missed part of the code that must be provided by you. 
\item Clean the workspace for ensuring that every file is compiled: \\
	\texttt{Menu > Project > Clean}
\end{enumerate}

Figure \figref{eclipse_project_structure} on the page \figpageref{eclipse_project_structure} gives an example of the structure of the SARL project that you should obtain.

\mfigure[p]{width=.33\linewidth}{eclipse_project_structure}{Example of the structure of a SARL project}{eclipse_project_structure}



\section{Brief Description of the Code Skeleton}

The skeleton contains a framework in the package \texttt{fr.utbm.info.vi51.framework}.
This framework contains the abstract implementation for the execution platform.
\emph{It is recommended to read this code and the associated Javadoc.}

The package \texttt{fr.utbm.info.vi51.labwork5} contains the code to complete during this lab work.

The subpackages are or will be:
\begin{itemize}
\item \texttt{fr.utbm.info.vi51.motionbehavior} is the package that contains the movement behaviors (kinematic and/or steering).
\item \texttt{fr.utbm.info.vi51.labwork4.environment} contains the definition of the environment and the objects inside that are specific to the lab work.
\item \texttt{fr.utbm.info.vi51.labwork4.gui} contains the UI for the project.
\item \texttt{fr.utbm.info.vi51.labwork4.agent} contains the code of the agent to complete.
\item The file \texttt{fr/utbm/info/vi51/labwork4/MainProgram.java} contains the main program.
\end{itemize}

\section{Work to be Done during the Lab Work Session}

The following sections describe the work to be done during this lab work session.

\subsection{Boid Behavior}

You must implement the behavior of a boid.

You should:
\begin{enumerate}[a)]
\item read the class \texttt{WorldModel} for understanding the types of objects that will be included in the agents' perceptions.
\item Write the behavior in \texttt{Fish} agent, for reproducing a Boid.
\end{enumerate}

\subsection{Environment Model for Ant Colony}

In this section, you must update the environment model for enabling the creation of pheromones in the environment, and their evaporation.

You should:
\begin{enumerate}[a)]
\item Create the type of environment object \texttt{Pheromone}.
\item Create the type of environment object \texttt{Food}.
\item Create a data structure for containing the pheronomons in the \texttt{WorldModel} class.
\item Update the \texttt{getAllObjects} function.
\item Update the \texttt{computeEndogenousBehaviorInfluences} function for evaporating the pheromons.
\item Update the \texttt{computePerceptionsFor} function for evaporating the pheromons.
\item Update the \texttt{applyInfluences} function for evaporating the pheromons.
\end{enumerate}

\subsection{Ant Behavior}

In this section, you must create the behavior of a ant that is searching for food.
You must use pheromone for returning to the colony, and/or returning to the food source.

You should:
\begin{enumerate}[a)]
\item Write the behavior in \texttt{Ant} agent, for reproducing an Ant.
\end{enumerate}

\subsection{Main Program for Ant Simulation}

You should:
\begin{enumerate}[a)]
\item Update the function \texttt{main} for launching ants.
\end{enumerate}

\end{document}
