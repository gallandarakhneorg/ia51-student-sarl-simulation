%----------------------------------------
% IT IS RECOMMENDED TO USE AUTOLATEX FOR
% COMPILING THIS DOCUMENT.
% http://www.arakhne.org/autolatex
%----------------------------------------

\documentclass[article,english,nodocumentinfo]{multiagentfrreport}

% The TeX code is entering with UTF8
% character encoding (Linux and MacOS standards)
\usepackage[utf8]{inputenc}
\usepackage{fancyhdr}
\usepackage{../common/sarl-listing}

\graphicspath{{imgs/auto/},{imgs/},{../common/}}

\declaredocument{VI51 Lab Work \#5}{Q-Learning}{UTBM-INFO-VI51-LW5}

\addauthorvalidator*[St\'ephane Galland]{St{\'e}phane}{Galland}{Teacher}

\updateversion{5.0}{\makedate{03}{05}{2015}}{First release on Github}{\upmpublic}

\Set{mafr_contact_name}{\phdname*{St\'ephane}{Galland}}
\Set{mafr_contact_email}{stephane.galland@utbm.fr}
\Set[french]{mafr_contact_phone}{03~84~58~34~18}
\Set[english]{mafr_contact_phone}{+33 384~583~418}

\gdef\skeletonName{\texttt{\mbox{LW5\_VI51\_skeleton\string.jar}}}

\begin{document}

\section{Goal of this Lab Work Session}

The goal of this lab work session is to write configure the provided Q-Learning algorithm for eanbling predators to catch a prey.

You shall learn: 
\begin{itemize}
\item How to confiure a Q-Learning algorithm.
\end{itemize}

\section{First Step: prepare the development environment}

In this section, you will find the tasks to do for preparing your development environment.

\subsection{Installation of the Eclipse Tools}

\begin{emphbox}
Recommended version of SARL : 0.3.1 \\
Recommended version of Janus : 2.0.3.1
\end{emphbox}

\begin{enumerate}
\item Download the \Emph{Eclipse product} that contains the compilation tools for the SARL programming language : \url{http://www.sarl.io}
\item Uncompress the Eclipse product for SARL.
\item Download the \Emph{development version} of the Janus agent execution platform : \url{http://www.janusproject.io}
\item Do not uncompress the Jar file of the Janus platform.
\item Launch the downloaded Eclipse product.
\item Open the wizard for creating a SARL project, with the menu: \\
	\texttt{> File > New > Project > SARL > Project}
\item Enter the name of the project.
\item Click on the tab with the name \code{Libraries}.
\item In the list of the libraries, remove the \code{SARL Libraries}.
\item In the list of the libraries, add the download file of the Janus platform as an external Jar file.
\item Click on "Finish", the SARL project should be created.
\item You must ensure that the configuration of your SARL project is correct (still a bug):
	\begin{enumerate}[a]
	\item Open the dialog of the properties of the SARL project by clicking on: \\
		\texttt{Right click on project > Properties > SARL > Compiler > Output Folder}
	\item Check if the field ``Directory'' is set to a source folder that is existing in your SARL project. If not change the property with): \\
		\texttt{src/main/generated-sources/xtend}
	\end{enumerate}
\end{enumerate}

You Eclipse is now ready for the lab work session. You should now install the code skeleton provided by the teachers.

\subsection{Installation of the Code Skeleton}

The teachers provide a code skeleton that should be completed by you for terminating the tasks related to this lab work session.
The steps to follow for installing the code skeleton are:
\begin{enumerate}
\item Download from the Moodle of UTBM (\url{http://moodle.utbm.fr}) the file with the name \skeletonName.
\item Open the wizard for importing the code skeleton into the SARL project: \\
	\texttt{Right click on project > Import > General > Archive File}
\item Select on the local file system the downloaded file of the code skeleton; and click on ``Finish''.
\item The source folders of your project shall contains SARL and Java code. \\
	Some errors are appearing since they are related to the missed part of the code that must be provided by you. 
\item Clean the workspace for ensuring that every file is compiled: \\
	\texttt{Menu > Project > Clean}
\end{enumerate}

Figure \figref{eclipse_project_structure} on the page \figpageref{eclipse_project_structure} gives an example of the structure of the SARL project that you should obtain.

\mfigure[p]{width=.33\linewidth}{eclipse_project_structure}{Example of the structure of a SARL project}{eclipse_project_structure}



\section{Brief Description of the Code Skeleton}

The skeleton contains a framework in the package \texttt{fr.utbm.info.vi51.framework}.
This framework contains the abstract implementation for the execution platform.
\emph{It is recommended to read this code and the associated Javadoc.}

The package \texttt{fr.utbm.info.vi51.labwork2} contains the code to complete during this lab work.

The subpackages are or will be:
\begin{itemize}
\item \texttt{fr.utbm.info.vi51.general.behavior} is the package that contains the movement behaviors (kinematic and/or steering).
\item \texttt{fr.utbm.info.vi51.general.qlearning} is the package in which the Q-Learning algorithm is implemented.
\item \texttt{fr.utbm.info.vi51.labwork5.environment} contains the definition of the environment and the objects inside that are specific to the lab work.
\item \texttt{fr.utbm.info.vi51.labwork5.gui} contains the UI for the project.
\item \texttt{fr.utbm.info.vi51.labwork5.agent} contains the code of the agent to complete.
\item The file \texttt{fr/utbm/info/vi51/labwork5/MainProgram.java} contains the main program.
\end{itemize}

\section{Principle of the Q-Learning}

Q-learning is a model-free reinforcement learning technique.
Specifically, Q-learning can be used to find an optimal action-selection policy for any given (finite) Markov decision process (MDP).
It works by learning an action-value function that ultimately gives the expected utility of taking a given action in a given state and following the optimal policy thereafter.
A policy is a rule that the agent follows in selecting actions, given the state it is in.
When such an action-value function is learned, the optimal policy can be constructed by simply selecting the action with the highest value in each state.
One of the strengths of Q-learning is that it is able to compare the expected utility of the available actions without requiring a model of the environment.
Additionally, Q-learning can handle problems with stochastic transitions and rewards, without requiring any adaptations.

The Q-learning algorithm is based on a graph of the states of the system.
An action permits to change the state.
Each time an action is selected, a feedback must be computed. This feedback determines if the action is ``good'' or ``bad''. This feedback is used for updating the Q-value of the (state, action) pair. The Q-value represents the interest for the action to be selected when the agent is in the given state. 

The provided Q-Learning algorithm needs to define several functions that are used as inputs.
They are defined below.

\section{Work to be Done during the Lab Work Session}

The following sections describe the work to be done during this lab work session.

The considered problem is: a carrot is moving on the environment (controled by the mouse) and predactors (agents) should learn to capture the carrot.

\subsection{Predator Actions}

First, you must define the types of actions that the predactors could do.
The following actions could be considered
\begin{lstlisting}
enum PredatorActionType {
   /** Move randomly. */
   RANDOM_MOVE,
   /** Move straight forward to the carrot. */
   MOVE_TO_KILL_THE_PREY,
   /** Move on the left side of the carrot. */
   MOVE_LEFT,
   /** Move on the right side of the carrot. */
   MOVE_RIGHT,
   /** Do nothing. */
   WAIT
}
\end{lstlisting}

Each action must be related to the Q-Learning algorithm.
You must defined a subclass of \code{QAction} that permits to describe the different types of actions.

In this subclass of \code{QAction}, several functions must be defined:
\begin{itemize}
\item \code{def toInt : int} \\
	replies the integer representation of the action.
\item \code{def compareTo(obj : QComparable) : int} \\
	replies the result of the comparison of the current action and the object given as parameter.
\end{itemize}

\subsection{Defining the states to explore}

The Q-learning algorithm works on a set of states.
You should define a set of states for the predator problem.
In other words, considering the state of the predactor body and of its perception, you should define the set of states that are needed for the learning process.

According to Q-learning approach, each state has a number.
In the \code{PredatorProblem} class, change the number of states to consider in the definition of the foeld with the name \code{states}.

Note that the semantic of each state (eg. ``carrot is straight forward'') should not be coded in a graph. This semantic is hard-coded in the functions you will code later.

\subsection{Available actions from a state}

It is needed to specify the actions that are available/activable from a specific state.

You must code the function \code{def getAvailableActionsFor(state : DefaultQState) : List<PredatorAction>} in the \code{PredatorProblem} class.

It is recommended to reply all the action for every state as a first version of this function.

\subsection{Find the Q-state from perception}

One important point in the Q-learning algorithm is to map the state of the simulated problem (the prey-predator problem) into a state in the Q-learning graph.

In the \code{PredatorProblem} class, the function \code{def translateCurrentState(position : Point2f, percepts : List<Percept>)} permits to change the current state (stored as a field in the \code{PredatorProblem} class) according to the position of the agent body and its perception.

\subsection{Computing the Feedback}

The feedback is the result of the evaluation of the execution of an action in a specific state.
You must code the \code{takeAction} function in the \code{PredatorProblem} class.

This function takes two parameters:
\begin{itemize}
\item the state in which the action is applied, and
\item the action to apply.
\end{itemize}

This function replies the result of the evaluation. Usually a negative score means a ``bad'' action, a positive score a ``good'' action, and a zero score an action that has no particular effect.

\subsection{Specifying the Algorithm Parameters}

Several parameters are used by the Q-Learning algorithm for controlling its behavior:
\begin{itemize}
\item $\alpha \in [0;1]$: The learning rate controls how must influence the current feedback value has over the stored Q-value. A value of zero means no-learning. A value of one would give no credence to any previous
experience.
\item $\gamma \in [0;1]$: The discount rate controls how must an action's Q-value depends on the Q-value at the state it leads to. A value of zero means what only the reward is taken into account, ie. no learning on sequences of actions A value of one would rate the reward for the current action as equally important as the quality of the state it leads to.
\item $\rho \in [0;1]$: The randomness for exploration controls how often the algorithm will take a random action, rather than the best it knows so far. A value of zero gives pure exploitation strategy: only the best
action is selected. A value of one gives pure exploration strategy: the algorithm will always try new things.
\item $\nu \in [0;1]$: The Length of walk controls the number of iterations that will be carried out in a sequence of connected actions. A zero value would mean the algorithm always uses the state it reached in the previous iteration as the starting state for the next iteration. A value of one would mean that every iteration starts from a random state.
\end{itemize}

You could change the values of these parameters (in the corresponding functions in the \code{PredatorProblem} class.

\end{document}
