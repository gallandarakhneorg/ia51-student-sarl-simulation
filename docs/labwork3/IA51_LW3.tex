%----------------------------------------
% IT IS RECOMMENDED TO USE AUTOLATEX FOR
% COMPILING THIS DOCUMENT.
% http://www.arakhne.org/autolatex
%----------------------------------------

\documentclass[article,english,nodocumentinfo]{multiagentfrreport}

% The TeX code is entering with UTF8
% character encoding (Linux and MacOS standards)
\usepackage[utf8]{inputenc}
\usepackage{fancyhdr}
\usepackage{../common/sarl-listing}

\graphicspath{{imgs/auto/},{imgs/},{../common/}}

\declaredocument{IA51 Lab Work \#3}{Coordinated Movement Algorithms}{UTBM-INFO-IA51-LW3}

\addauthorvalidator*[St\'ephane Galland]{St{\'e}phane}{Galland}{Teacher}

\updateversion{5.0}{\makedate{03}{05}{2015}}{First release on Github}{\upmpublic}
\incversion{\makedate{29}{05}{2017}}{Move to IA51}{\upmpublic}

\Set{mafr_contact_name}{\phdname*{St\'ephane}{Galland}}
\Set{mafr_contact_email}{stephane.galland@utbm.fr}
\Set[french]{mafr_contact_phone}{03~84~58~34~18}
\Set[english]{mafr_contact_phone}{+33 384~583~418}

\gdef\skeletonName{\texttt{\mbox{LW3\_IA51\_skeleton\string.jar}}}

\begin{document}

\section{Goal of this Lab Work Session}

The goal of this lab work session is to write a simulator in which agents are able to move in a coordinated way.
The work should be done in the programming languages SARL (\url{http://www.sarl.io}) and Java.
You shall use code skeleton provided by the tearchers with the simulation framework already coded inside.

You shall learn: 
\begin{itemize}
\item How to write a general API for the formation patterns.
\item How to write one or more specific formation patterns.
\item How to put a formation pattern in the knowledge of the agents.
\item How to use the formation pattern for determining the targets of the movements of the agents. 
\end{itemize}

\section{First Step: prepare the development environment}

In this section, you will find the tasks to do for preparing your development environment.

\subsection{Installation of the Eclipse Tools}

\begin{emphbox}
Recommended version of SARL : \sarlversion
\end{emphbox}

\begin{enumerate}
\item Download the \Emph{Eclipse product} that contains the compilation tools for the SARL programming language : \url{http://www.sarl.io}
\item Uncompress the Eclipse product for SARL.
\item Launch the downloaded Eclipse product.
\item Open the wizard for creating a SARL project, with the menu: \\
	\texttt{> File > New > Project > SARL > Project}
\item Enter the name of the project.
%\item Click on the tab with the name \code{Libraries}.
%\item \textcolor{red}{In the list of the libraries, remove the \code{SARL Libraries}.}
%\item \textcolor{red}{In the list of the libraries, add the \code{Janus Libraries}} (the skeletons that are provided for this lab work has a dependency to an class that is provided by the Janus libraries).
\item Click on "Finish", the SARL project should be created.
%\item You must ensure that the configuration of your SARL project is correct:
%	\begin{enumerate}[a]
%	\item Open the dialog of the properties of the SARL project by clicking on: \\
%		\texttt{Right click on project > Properties > SARL > Compiler > Output Folder}
%	\item Check if the field ``Directory'' is set to a source folder that is existing in your SARL project. If not change the property with): \\
%		\texttt{src/main/generated-sources/sarl}
%	\end{enumerate}
\end{enumerate}

Your Eclipse is now ready for the lab work session. You should now install the code skeleton provided by the teachers.

\subsection{Installation of the Code Skeleton}

The teachers provide a code skeleton that should be completed by you for terminating the tasks related to this lab work session.
The steps to follow for installing the code skeleton are:
\begin{enumerate}
\item Download from the Teams of UTBM the file with the name \skeletonName.
\item Open the wizard for importing the code skeleton into the SARL project (right-click on the project name in the package explorer)	: \\
	\texttt{Right click on project > Import > General > Archive File}
\item Select on the local file system the downloaded file of the code skeleton; and click on ``Finish''.
\item The source folders of your project shall contains SARL and Java code. \\
	Some errors are appearing since they are related to the missed part of the code that must be provided by you. 
\item Clean the workspace for ensuring that every file is compiled: \\
	\texttt{Menu > Project > Clean}
\end{enumerate}

Figure \figref{eclipse_project_structure} on the page \figpageref{eclipse_project_structure} gives an example of the structure of the SARL project that you should obtain.

\mfigure[p]{width=.33\linewidth}{eclipse_project_structure}{Example of the structure of a SARL project}{eclipse_project_structure}



\section{Brief Description of the Code Skeleton}

The skeleton contains a framework in the package \texttt{fr.utbm.info.IA51.framework}.
This framework contains the abstract implementation for the execution platform.
\emph{It is recommended to read this code and the associated Javadoc.}

The package \texttt{fr.utbm.info.IA51.labwork4} contains the code to complete during this lab work.

The subpackages are or will be:
\begin{itemize}
\item \texttt{fr.utbm.info.IA51.motionbehavior} is the package that contains the movement behaviors (kinematic and/or steering).
\item \texttt{fr.utbm.info.IA51.formation} is the package in which you must create the formation patterns.
\item \texttt{fr.utbm.info.IA51.labwork4.environment} contains the definition of the environment and the objects inside that are specific to the lab work.
\item \texttt{fr.utbm.info.IA51.labwork4.gui} contains the UI for the project.
\item \texttt{fr.utbm.info.IA51.labwork4.agent} contains the code of the agent to complete.
\item The file \texttt{fr/utbm/info/IA51/labwork4/MainProgram.java} contains the main program.
\end{itemize}

\section{Work to be Done during the Lab Work Session}

The following sections describe the work to be done during this lab work session.

\subsection{API for the Formation}

You must implement a data structure able to represent a formation: the formation object.
Figure \figref{classdiagram} provides you an example of UML class diagram of the formation API.

\mfigure[p]{width=\linewidth}{classdiagram}{Class diagram of the formation API}{classdiagram}

You should:
\begin{enumerate}[a)]
\item create the abstract class \texttt{Anchor}, representing a position in the formation. This position could be for the leader or for the following agents.
\item create the class \texttt{Formation}, representing the global pattern of the formation.
\end{enumerate}

\subsection{Body Guard Formation Pattern}

You must create a specific implementation of a formation pattern: the body guard (or circle) formation.
In this pattern, the leader is at the center of formation.
The following agents are the body guards: they are located on circle centered on the leader of the formation, and they see outside the formation (see Figure \figref{bodyguardpattern}, page \figpageref{bodyguardpattern}).
The definition of the formation pattern must be coded into the function \texttt{scale(int)}, where the parameter is the number of anchors in the pattern (including the leader anchor).

\mfigure[p]{width=.6\linewidth}{bodyguardpattern}{The body guard formation pattern}{bodyguardpattern}

\begin{upmcaution}
\Emph{Caution:} It is important to have a clear distinction between the concepts of ``formation anchor'', ``agent'', and ``agent body''. \newline\newline
The ``agent'' is the artificial intelligence which is exhibiting a behavior. It is intelligent and autonomous. \newline\newline
The ``agent body'' is the representation of the agent on the physic world. It is not intelligent nor autonomous. \newline\newline
The ``formation anchor'' is a representation of a position in a formation pattern. It is not an agent nor an agent body.
\end{upmcaution}

\subsection{Leader Behavior}

Assuming that the code skeleton puts the formation pattern into the knowledge of the leader agent, write the behavior of the leader agent.
The leader agent is running randomly on the ground, and controls the formation pattern.
Figure \figref{bodyguardagents} on page \figpageref{bodyguardagents} provides an example of relationship between the leader body and the body guard formation pattern.

\mfigure[p]{width=.5\linewidth}{bodyguardagents}{The agent bodies related to the body guard formation pattern}{bodyguardagents}

You shall:
\begin{enumerate}[a)]
\item use the movement behaviors in the SARL leader agent, in \code{on PerceptionEvent}.
\item update the previous event handler for updating the position and orientation of the formation pattern.
\end{enumerate}

\subsection{anchor Assignment Policy}

Since the number of anchors and the number of agents may not match, it is important to define a policy for assigning the anchors to the agents.
This policy should be coded into a subclass of \texttt{FormationAssignmentPolicy}.
An instance of this class is provided to each following agent for obtaining a anchor.

The function \code{assignAnchorToMe()} is invoked by the following agents for obtaining the anchor that should be used.

You shall code this class.

\subsection{Follower Behavior}

Assuming that the code skeleton puts the formation pattern into the knowledge of the follower agent, write the behavior of the following agent.
The following agent is seeking the position of the assigned anchor in the formation, and aligning to the anchor's orientation.
Figure \figref{bodyguardagents} on page \figpageref{bodyguardagents} provides an example of relationship between the follower bodies and the body guard formation pattern.

You shall:
\begin{enumerate}[a)]
\item write the code for retreiving the position and orientation of the formation's anchor in the SARL follower agent.
\item implement the movement behaviour of the SARL follower agent.
\end{enumerate}

\end{document}
