%----------------------------------------
% IT IS RECOMMENDED TO USE AUTOLATEX FOR
% COMPILING THIS DOCUMENT.
% http://www.arakhne.org/autolatex
%----------------------------------------

\documentclass[article,english,nodocumentinfo]{multiagentfrreport}

% The TeX code is entering with UTF8
% character encoding (Linux and MacOS standards)
\usepackage[utf8]{inputenc}
\usepackage{fancyhdr}
\usepackage{../common/sarl-colorized-listing}

\graphicspath{{imgs/auto/},{imgs/},{../common/}}

\declaredocument{VI51 Lab Work \#2}{Environment Model}{UTBM-INFO-VI51-LW2}

\addauthorvalidator*[St\'ephane Galland]{St{\'e}phane}{Galland}{Teacher}

\updateversion{5.0}{\makedate{03}{05}{2015}}{First release on Github}{\upmpublic}

\Set{mafr_contact_name}{\phdname*{St\'ephane}{Galland}}
\Set{mafr_contact_email}{stephane.galland@utbm.fr}
\Set[french]{mafr_contact_phone}{03~84~58~34~18}
\Set[english]{mafr_contact_phone}{+33 384~583~418}

\gdef\skeletonName{\texttt{\mbox{LW2\_VI51\_skeleton\string.jar}}}

\begin{document}

\section{Goal of this Lab Work Session}

The goal of this lab work session is to write a spatial tree for a simulation environment.

You shall learn: 
\begin{itemize}
\item How to write a general API for spatial trees.
\item How to extract information from the tree for the agents.
\item How to apply updates on the tree.
\end{itemize}

\section{First Step: prepare the development environment}

In this section, you will find the tasks to do for preparing your development environment.

\subsection{Installation of the Eclipse Tools}

\begin{emphbox}
Recommended version of SARL : 0.3.1 \\
Recommended version of Janus : 2.0.3.1
\end{emphbox}

\begin{enumerate}
\item Download the \Emph{Eclipse product} that contains the compilation tools for the SARL programming language : \url{http://www.sarl.io}
\item Uncompress the Eclipse product for SARL.
\item Download the \Emph{development version} of the Janus agent execution platform : \url{http://www.janusproject.io}
\item Do not uncompress the Jar file of the Janus platform.
\item Launch the downloaded Eclipse product.
\item Open the wizard for creating a SARL project, with the menu: \\
	\texttt{> File > New > Project > SARL > Project}
\item Enter the name of the project.
\item Click on the tab with the name \code{Libraries}.
\item In the list of the libraries, remove the \code{SARL Libraries}.
\item In the list of the libraries, add the download file of the Janus platform as an external Jar file.
\item Click on "Finish", the SARL project should be created.
\item You must ensure that the configuration of your SARL project is correct (still a bug):
	\begin{enumerate}[a]
	\item Open the dialog of the properties of the SARL project by clicking on: \\
		\texttt{Right click on project > Properties > SARL > Compiler > Output Folder}
	\item Check if the field ``Directory'' is set to a source folder that is existing in your SARL project. If not change the property with): \\
		\texttt{src/main/generated-sources/xtend}
	\end{enumerate}
\end{enumerate}

You Eclipse is now ready for the lab work session. You should now install the code skeleton provided by the teachers.

\subsection{Installation of the Code Skeleton}

The teachers provide a code skeleton that should be completed by you for terminating the tasks related to this lab work session.
The steps to follow for installing the code skeleton are:
\begin{enumerate}
\item Download from the Moodle of UTBM (\url{http://moodle.utbm.fr}) the file with the name \skeletonName.
\item Open the wizard for importing the code skeleton into the SARL project: \\
	\texttt{Right click on project > Import > General > Archive File}
\item Select on the local file system the downloaded file of the code skeleton; and click on ``Finish''.
\item The source folders of your project shall contains SARL and Java code. \\
	Some errors are appearing since they are related to the missed part of the code that must be provided by you. 
\item Clean the workspace for ensuring that every file is compiled: \\
	\texttt{Menu > Project > Clean}
\end{enumerate}

Figure \figref{eclipse_project_structure} on the page \figpageref{eclipse_project_structure} gives an example of the structure of the SARL project that you should obtain.

\mfigure[p]{width=.33\linewidth}{eclipse_project_structure}{Example of the structure of a SARL project}{eclipse_project_structure}



\section{Brief Description of the Code Skeleton}

The skeleton contains a framework in the package \texttt{fr.utbm.info.vi51.framework}.
This framework contains the abstract implementation for the execution platform.
\emph{It is recommended to read this code and the associated Javadoc.}

The package \texttt{fr.utbm.info.vi51.labwork2} contains the code to complete during this lab work.

The subpackages are or will be:
\begin{itemize}
\item \texttt{fr.utbm.info.vi51.general.behavior} is the package in which you must create the behaviors (kinematic and/or steering).
\item \texttt{fr.utbm.info.vi51.labwork2.environment} contains the definition of the environment and the objects inside that are specific to the lab work.
\item \texttt{fr.utbm.info.vi51.labwork2.gui} contains the UI for the project.
\item \texttt{fr.utbm.info.vi51.labwork2.agent} contains the code of the agent to complete.
\item The file \texttt{fr/utbm/info/vi51/labwork2/MainProgram.java} contains the main program.
\end{itemize}

\section{Work to be Done during the Lab Work Session}

The following sections describe the work to be done during this lab work session.

\subsection{Tree Implementation}

Write the \code{Tree} and \code{TreeNode} classes.
Each node must contains a shape that represents the covered space.
Each node may contains a list of objects.

\subsection{Perception Iterator}

Write the \code{FrustumIterator} that replies the objects which are stored in a spatial tree and under intersection with a frustum.

\subsection{Include Tree in the WorldModel}

Update the \code{WorldModel} class:
\begin{itemize}
\item Add the tree in the code.
\item Update the \code{onAgenBodyCreated} and \code{onAgentNodyDestroyed} functions.
\item Update the \code{computePerceptionFor} function.
\item Update the \code{applyInfluences} function.
\end{itemize}

\end{document}
